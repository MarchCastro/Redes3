\subsection{Castro Flores Marcela}
Creo que el desarrollo de esta práctica fue un poco difícil para mi en primera instancia porque tuve que familiarizarme rápidamente con el lenguaje de programación Python mismo que si bien no es muy difícil, presenta ciertos cambios al momento de manejar ciertas funciones. Por otro lado, nunca antes había utilizado el entorno gráfico de Python, es por ello que me tomó más tiempo del normal realizar ciertas partes de las secciones que me tocaron. 
Creo que la utilización de sistemas más complejos, pero semejantes al que se realizó como práctica resultan ser bastante útiles para reportar y mejorar las áreas en las que los dispositivos que crean diferentes topologías de red llegan a tener pequeños problemas que alentan las peticiones que los clientes realizan. 
\subsection{Sánchez Cruz Rosa María}
La implemetación de la practica fué útil para determinar cómo funciona un agente, un gestor y las diferencias entre ambos.
Un agente lo podemos determinar como aquel equipo dentro de una red que se tiene la necesidad de saber algunas características, por ejemplo conexione, rendimiento,  Información.
El sistema gestor nos ayuda a recibir e interpretar los datos que recibimos del agente, este sistema gestor lo podemos asociar como un administrador el cual puede adquirir la información de los equipos agentes, con el objetivo principal de tener un control de los servicios o como prevención de incidentes.
La forma que captura la información o el medio por el cual se recibe la información es a través del protocolo SNMP que por medio de OIDs que son valores prestablecidos y estructurados es posible obtener una respuesta con la información que se ha pedido. Es necesario añadir la dirección ip y una comunidad con la cual el agente esta asociado.
Con respecto a la práctica se desarrolló un sistema gestor el cual puede administrar equipos con las funcionalidades de agregar y eliminar agentes con la característica de que cada agente tiene asociado cinco capturas SNMP que se realizan de manera concurrente y los resultados son graficados y presentados por medio de la libreria RRDTool. Es importante la aplicación desarrollada por el hecho de aprender a diferenciar agentes y gestores. La característica más importante a considerar es el medio de comunicación y saber interpretar la información que podemos obtener con SNMP ya que es la base del desarrollo de prácticas complejas y un sin número de aplicaciones.

\subsection{Santiago Mancera Arturo Samuel}
La realización de la presente práctica fue de gran utilidad para comprender el funcionamiento de algunas herramientas comerciales como Nagios o Cacti. Realizar desde cero un programa con funciones similares ayuda a comprender la complejidad de su estructura.
La parte más importante, personalmente, fue la de la adquisición de información mediante SNMP y su posterior procesamiento para la construcción de gráficas. Es importante mencionar que se comprendió la utilidad del uso de bases de datos de tipo \textit{Round Robin} en la persistencia de datos correspondientes al monitoreo.
Por otro lado, el uso de Python facilita, en gran medida, el desarrollo de la herramienta en los aspectos de funcionalidad e interfaz con el usuario.