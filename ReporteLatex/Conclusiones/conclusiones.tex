\subsection{Castro Flores Marcela}
Creo que el desarrollo de esta práctica fue un poco difícil para mi en primera instancia porque tuve que familiarizarme rápidamente con el lenguaje de programación Python mismo que si bien no es muy difícil, presenta ciertos cambios al momento de manejar ciertas funciones. Por otro lado, nunca antes había utilizado el entorno gráfico de Python, es por ello que me tomó más tiempo del normal realizar ciertas partes de las secciones que me tocaron. 
Creo que la utilización de sistemas más complejos, pero semejantes al que se realizó como práctica resultan ser bastante útiles para reportar y mejorar las áreas en las que los dispositivos que crean diferentes topologías de red llegan a tener pequeños problemas que alentan las peticiones que los clientes realizan. 
\subsection{Sánchez Cruz Rosa María}
\subsection{Santiago Mancera Arturo Samuel}
La realización de la presente práctica fue de gran utilidad para comprender el funcionamiento de algunas herramientas comerciales como Nagios o Cacti. Realizar desde cero un programa con funciones similares ayuda a comprender la complejidad de su estructura.
La parte más importante, personalmente, fue la de la adquisición de información mediante SNMP y su posterior procesamiento para la construcción de gráficas. Es importante mencionar que se comprendió la utilidad del uso de bases de datos de tipo \textit{Round Robin} en la persistencia de datos correspondientes al monitoreo.
Por otro lado, el uso de Python facilita, en gran medida, el desarrollo de la herramienta en los aspectos de funcionalidad e interfaz con el usuario.