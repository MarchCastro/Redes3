Para la realización de esta práctica se utilizaró nuevamente el protocolo SNMP, sin embargo esta vez se implementaron 3 algoritmos que serán explicados posteriormente.\\
Dichos algoritmos son:
\begin{itemize}
\item Línea de base
\item Mínimos cuadrados
\item Holt Winters
\end{itemize} 

Esta práctica se dividió en las tres partes siguientes:
\begin{enumerate}
\item EXPLIQUEN RAPIDO SU PARTE
\item La segunda parte es el algoritmo de Mínimos Cuadrados, el cual nos da una línea recta que es el resultado de una predicción con base a una colección de datos que nso pretende proyectar el crecimeinto de los mismos,en este caso nuestra colección de datos se basa en el uso de la CPU.
\item Por último, la tercera parte correspondió al algoritmo de Holt Winters el cual se encargaba de identificar dentro de una gráfica no lineal, si los valores medidos salían de cierto rango de medición tanto superior como inferior y cuando esto sucedía, se notificaba por medio de un correo electrónico al administrador de la red.
\end{enumerate}

En el capítulo mostrado a continuación se observa el desarrollo de la práctica.



 