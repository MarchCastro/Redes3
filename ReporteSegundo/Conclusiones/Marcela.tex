\begin{itemize}
\item \textbf{Castro Flores Marcela}
\end{itemize}

La realización de esta segunda práctica me pareció muy interesante porque fue una integración de lo que ya se había implementado en el primer parcial más las nuevas funcionalidades tales como notificar al administrador cuando un fallo ha ocurrido.
Pienso que es importante el conocer este tipo de funcionalidades pues nos permite optimizar los servicios de las redes y también prevenir la situación en la cual un fallo pueda ocasionar un retraso en un proyecto por ejemplo pues esto podría ocasionar un retraso y un incremento en costos.
\\ \par
Así mismo, creo que es interesante el aprender nuevos algoritmos como los aplicados en este parcial ya que nos muestra una perspectiva de como aunque parezca que las matemáticas, la probabilidad y las redes informáticas pertenecen a áreas diferentes, en conjunto permiten la implementación de alternativas que evitan o minimizan el impacto de un fallo.