\begin{itemize}
\item \textbf{Castro Flores Marcela}
\end{itemize}

El desarrollo de esta práctica fue un poco difícil debido a las múltiples formas en las cuales se pueden configurar las topologías y de las cuales se obtenían los archivos de configuración de los diferentes routers, mismos que posteriormente, podían ser importados nuevamente a ellos con ciertas modificaciones y funcionalidades. Sin embargo, al realizar la implementación de las diferentes librerías de python tanto para realizar el monitoreo de los diferentes servidores, como también para obtener los archivos de configuración mencionados anteriormente, me di cuenta que era mucho más sencilla y mucho más rápida la obtención de la información como los tiempos de respuesta o bytes transferidos y también se optimizaban muchos de los pasos requeridos para simplemente obtener un archivo.\\ \
Creo que la implementación de este tipo de topologías nos acerca a una visión más próxima de como están realmente configuradas y conectadas las grandes redes tanto a nivel local como a un nivel más industrial.

