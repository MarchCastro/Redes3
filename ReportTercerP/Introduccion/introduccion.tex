Para la realización de la última práctica se implementaron diferentes servidores con los cuales por medio de la ejecución de diferentes sensores se obtenían tanto sus diferentes tiempos de ejecución como las diferentes respuestas que el servidor desplegaba según era el caso.\\
Los servidores implementados son los siguientes:
\begin{itemize}
\item Servidor de correo electrónico (SMTP)
\item Servidor web (HTTP)
\item Servidor de archivos (FTP/FTP Server File Count)
\item Servidor de impresión (SNMP)
\item Servidor de acceso remoto (SSH)
\end{itemize} 

Esta práctica se dividió en las tres partes siguientes:
\begin{enumerate}
\item En la primera parte se requirió realizar la instalación y la configuración de los distintos servidores mencionados anteriormente mismos que fueron separados en diferentes máquinas virtuales con el fin de que en el paso posterior se pudiera dividir la topología dada.
\item La segunda parte fue la configuración de la topología que iba desde realizar la conexión correcta entre los dispositivos como también realizar las diferentes comprobaciones para verificar que si habia una conexión entre todos los dispositivos.
\item Por último, la tercera parte correspondió a la utilización de los diferentes sensores que verificaban la correcta respuesta y funcionamiento de cada servidor y de igual manera, se utilizó el administrador de archivos de configuración con el cual se tenia tanto la posibilidad de descargar como de subir los archivos correspondientes a cada uno de los routers disponibles.
\end{enumerate}

En el capítulo mostrado a continuación se observa el desarrollo de la práctica indicando tanto el código utilizado para la implementación de los sensores de cada servidor, como las pantallas que muestran paso a paso el proceso que se realizó para la configuración de la topología dada.



 